\chapter{Introduction}
\label{ch:intro}
In July 2011, I spontaneously started writing ``A trip through the Graphics Pipeline 2011'', a series of
blog posts that would eventually number 13~parts and 35000~words. The goal was to explain the nuts and bolts
of the graphics pipeline as implemented in modern high-end (desktop) GPUs. The only planning that went into it
was a two-page outline that contained nothing more than a rough breakdown into separate posts, with
maybe four or five keywords per part. I wrote the parts in sequence and posted them as soon as they were
ready---initially at a rate of one per day, but slowing down quite a bit towards the end. The big advantage
was that I got immediate feedback, but it also meant that the posts were all essentially in draft state and
never got properly revised. There were also some holes and structural problems with the original outline that
I was stuck with. This text is the attempt to rectify that situation by revising and expanding upon the original
material and collecting it all in one place.

The intended audience is programmers with an interest in the low-level workings of modern GPUs. Familiarity with
the basic ideas of modern CPU architectures (caches, pipelining, SIMD code, multi-core architectures) is assumed,
and a basic understanding of how hardware in general works doesn't hurt.

This text is broken down into chapters, roughly corresponding to the different topics covered in the original
series of blog posts. The chapters are fairly self-contained and initially proceed in ``dataflow order''---from the
application generating draw calls all the day down to pixels being written to the frame buffer, explaining the most
commonly used path (Vertex and Pixel Shaders, but none of the other shader types) first. Later chapters cover Geometry
Shaders and the Tessellation pipeline, both of which extend the basic graphics pipeline with more front-end
(pre-rasterization) stages.
